\section{Existing solutions}

We have done some investigations to other solutions found on the internet. An interesting and very relevant product was a Facebook like indicator that would glow when something you posted was liked. The interesting was that this was done with an Arduino board inside a Lego hand. Another is a mailbox rising it’s arm whenever you got a notification on Facebook. The relevancy for us is that we see it is possible to connect with Facebook in a tangible way using the Arduino. Our own ideas to not have to differ much from already existing concepts, our goal is to show that we can develop a general concept that would encompass these ideas if that was the case. We also looked into other social networks than Facebook like Twitter or LinkedIn. The advantage of other social networks than Facebook is that most use the open source social application API called OpenSocial. Facebook is arguably one of the most popular networks out there, but uses a proprietary API that is not compatible with all the other social networks.

Ultimately we presented various ideas and solutions to the customer and let the customer decide on what they would like to see. As a prototype he would like to see a t-shirt with various signals like LEDS that lighted whenever the user had status updates on Facebook, Twitter or another social network. Major advantage of this solution we can prototype connection to multiple social networks at the same time, so it is a good proof of concept for a tangiable Arduino device connected to multiple social networks.