\section{First approach}

Our first approach to design the system architecture was to develop:

\begin{itemize}
\item A software framework that
abstracted many common concepts found in different social networks (Facebook, Twitter, LinkedIn..)
like User, Post, Reply, Image and so on, to give the developers the possibility to use
(and extend) these abstractions seamlessly between different social networks.
\end{itemize}

\begin{itemize}
\item A java library to connect, using different mediums, to the Arduino devices in order
to send the data fetched from the social networks, and also a mechanism to manage Arduino's firmware,
to allow the device to support different configurations.
\end{itemize}

After the second meeting with the customer we decided to revise this plan, as this was not what
our customer really had in mind. Some of the code and documentation we produced earlier could be re-used, but some could not.

See appendix~\ref{app:firstdraftarchi} for more.

\section{Revised approach}
The biggest difference in the new design is that API will still provide abstractions
for concepts common to the social network, but it will now focus on the estabilishing an interface,
using Android's Intent mechanism, between Social applications and the Bluetooth library.

\begin{figure}[h!]
\centering \includegraphics[scale=0.35]{img/architecture-toplevel.png}
\caption{Revised architecture}
\label{fig:architecture}
\end{figure}

\newpage
\section{Architecture diagram}
The API layer is the social service layer. It uses androids build in broadcast intents to send and recive data to social services. Included in the social service we have classes that extends Intent class to indicate if its a social Post, social Group, social Image, social User etc. Multiple apps can use the API layer to send and retrive this data. 

Example:
The API layer is connected with twitter. Twitter retrives a new post and the API layer sends out a broadcast intent marked as $SOCIAL_POST_IN$. The android OS see who are listeneing to $SOCIAL_POST_IN$ and relay the message to all connected apps. The Developed app now has the post. If the developed app wants to send a post it sends an intent with $SOCIAL_POST_OUT$ that our API can pick up.

The bluetooth protocol in the application is a framework that the developed application includes to be able to send and recive raw data from the arduino board. It includes handshake and confirmation commands that is similar to tcp.

The arduino board includes a wrapper around the input/output of the firmware to connect to our protocol. 
\begin{figure}[hb!]
\centering \includegraphics[scale=0.40]{img/architecture-diagram.png}
\caption{System architecture}
\label{fig:architecture}
\end{figure}

\newpage
\section{Use cases}
Use Case:
We have two actors we are working with. One is the developer creating the application using our framework. The other one is the end user with the final product. We are not directly working with the end user as an actor, but we still want to make our framework so it is simple for developer to have a product with an easy connection to the arduino board.

Our focus on the actor developer is centered around the software he create. We do not know, and we don't need to know what product he is making. We will work on two parts he can to implement in his software. It will be the setup and transfer protocol to arduino and Intents to send and recive updates from social services.

For the end user, we want the setup to connect his phone to the developer to be as easy as possible. The focus will be on simplified connection to arduino board, and easy connection to social services from application.
\begin{figure}[hb!]
\centering \includegraphics[scale=0.50]{img/use-cases.png}
\caption{Use cases}
\label{fig:architecture}
\end{figure}