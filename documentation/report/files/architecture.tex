\section{First approach}

Our first approach to design the system architecture was to develop:

\begin{itemize}
\item A software framework that
abstracted many common concepts found in different social networks (Facebook, Twitter, LinkedIn..)
like User, Post, Reply, Image and so on, to give the developers the possibility to use
(and extend) these abstractions seamlessly between different social networks.

\item A java library to connect, using different interfaces, to the Arduino devices in order
to send the data fetched from the social networks, and also a mechanism to manage Arduino's firmware,
to allow the device to support different configurations.
\end{itemize}

After the second meeting with the customer we decided to revise this plan, as this was not what
our customer really had in mind. Some of the code and documentation we produced earlier could be re-used, but some could not.

See appendix~\ref{app:firstdraftarchi} for more.

\newpage

\section{Revised approach}
The biggest difference in the new design is that API will still provide abstractions
for concepts common to the social network, but it will now focus on estabilishing an interface,
using Android's Intent mechanism, between Social applications and the Android-Arduino applications.

\begin{figure}[h!]
\centering \includegraphics[scale=0.35]{img/architecture-toplevel.png}
\caption{Revised architecture}
\label{fig:architecture-toplevel}
\end{figure}

\newpage

\subsection{First design}
Our first design was based on the scenario where the user would download the t-shirt
and the social applications, browse for some content from within the social application
and actively send it to the t-shirt application pressing a 'share' button.
This scenario is represented in the following use case.

\begin{figure}[h!]
\centering \includegraphics[scale=0.35]{img/use-case1.png}
\caption{Use case for the first design}
\label{fig:use-case1}
\end{figure}

Different sample applications following this design were developed and showed to the
customer for feedback, and this helped us realize at some point that this design wouldn't fit
the customer ideas for the final product.

\subsection{Second design}
During the meetings with the customer, in which we showed him various simple prototype applications
in order to receive feedback and as much information we could use to proceed with the system design,
we understood that our first design was not really fitting in the common use-case scenarios
that the customer came up with. This led to a different design.



\newpage
\section{Architecture diagram}
The API layer is the social service layer. It uses androids build in broadcast intents to send and
recive data to social services. Included in the social service we have classes that extends Intent
class to indicate if its a social Post, social Group, social Image, social User etc. Multiple apps
can use the API layer to send and retrive this data.

Example:
The API layer is connected with twitter. Twitter retrives a new post and the API layer sends out
a broadcast intent marked as $SOCIAL_POST_IN$. The android OS see who are listeneing to $SOCIAL_POST_IN$ and relay the message to all connected apps. The Developed app now has the post. If the developed app wants to send a post it sends an intent with $SOCIAL_POST_OUT$ that our API can pick up.

The bluetooth protocol in the application is a framework that the developed application includes
to be able to send and recive raw data from the arduino board. It includes handshake and
confirmation commands that is similar to tcp.

The arduino board includes a wrapper around the input/output of the firmware to connect to our protocol.
\begin{figure}[hb!]
\centering \includegraphics[scale=0.40]{img/architecture-diagram.png}
\caption{System architecture}
\label{fig:architecture-diagram}
\end{figure}

\section{Package Structure}
The system can be divided into smaller sections. This section describes some of the structures
we have divided the system into. The most obvious distinction is between the Java code runs
on the Android platform and the C code which is executed by the Arduino platform.
The Java code is again divided into packaged libraries as described below.

\subsection{Java Code}
\begin{itemize}
\item{no.ntnu.osnap.com}\newline
Communication Library contains every class to establish a connection with a remote device using a general protocol interface. The actual details of the communication such as
Bluetooth, cable, WiFi, stream-based or sockets are hidden away from the user. This is to provide a simple and generic interface to all supported communication methods. The
user can easily add new communication types by extending an abstract class.
\item{no.ntnu.osnap.com.testing}\newline
Test units and sample programs for the ComLib. These are simple test applications that are run on the Android to test if the ComLib is communication correctly using the specified
protocol.
\item{no.ntnu.osnap.social}\newline
Social Library provides an easy to use interface between social networks such as Facebook, Twitter and Myspace and any application or service running on the Android platform.
The SocialLib also provides some general data models such as Person, Group or Post which are concepts that are similar between any social network. 
\item{no.ntnu.osnap.social.testing}  \newline
Test units and sample programs for the SocialLib.
\end{itemize}

\subsection{Arduino C code}
C source code for the Arduino firmware supporting the ComLib protocol.

\newpage
\section{Use cases}
Use Case:
We have two actors we are working with. One is the developer creating the application using our framework. The other one is the end user with the final product. We are not directly working with the end user as an actor, but we still want to make our framework so it is simple for developer to have a product with an easy connection to the arduino board.

Our focus on the actor developer is centered around the software he create. We do not know, and we don't need to know what product he is making. We will work on two parts he can to implement in his software. It will be the setup and transfer protocol to arduino and Intents to send and recive updates from social services.

For the end user, we want the setup to connect his phone to the developer to be as easy as possible. The focus will be on simplified connection to arduino board, and easy connection to social services from application.
\begin{figure}[hb!]
\centering \includegraphics[scale=0.50]{img/use-cases.png}
\caption{Use cases}
\label{fig:architecture-usecases}
\end{figure}

