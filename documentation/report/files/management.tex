
\section{Development Process}

Given the iterative nature of our work schedule and of the meeting
with our customer, as well as his request for intermediate prototypes
to be presented during these meetings, the choice of Scrum as a development
process was natural. Moreover, everybody in the team had previous
experiences with Scrum from earlier projects. With our choice of Scrum
as process development model, Some changes to the model were adopted so
that we could follow the intended process as closely as our task and experience allow.
This changes are described in the following section of this document.


\subsection{Modifications to the development process}

To make it more suitable to our schedule we made some slight modifications
to the Scrum model. The group meetings will be held three times a week (on
Monday, Wednesday and Friday) and we will have extended work sessions on these
days too. Sprints will be modified to be primarly bi-weekly, but time
between Sprints may vary depending on the priorities identified during
the meetings with the customer and on the risk analysis. Sprint backlogs
will be very crucial in this regard. Once the goals are set,
the gradual inclusion of new features as the project evolves will
be documented in the backlogs from sprint to sprint.
A product backlog with feedback provided by the customer will also be produced.
The backlogs will be modified encompass both requirements and feature goals.
Another important difference from the normal Scrum process is that the
specifications of the prototype that will be produced will be defined
little by little in agreement with our customer during the weekly meetings.

\section{Project management tools}

To accompany us in the Scrum process we have chosen an online tool
called ScrumDo (www.scrumdo.com). This tool has features for most
if not all parts of the Scrum process. Using this tool consistently
will be our main method of maintaining and separating packages from
the WBS (Work Breakdown Structure). In the context of ScrumDo and
Scrum these low level work packages are called stories and are moved
accordingly from \char`\"{}ToDo\char`\"{} over to \char`\"{}In progress\char`\"{}
and eventually to \char`\"{}Done'' areas.
	
\begin{figure}[h!]
\centering \includegraphics{img/management-scrumdo} \caption{The Scrum Board in ScrumDo}

\label{fig:management-scrumdo}
\end{figure}
	
Another part of the project is the collaboration software.
Initially we decided to use Git which is a fast, scalable,
distributed revision control system with an unusually rich command
set that provides both high-level operations and full access to internals.
It differs from the centralised control systems like SVN for its capability
to handle several software \char`\"{}branches\char`\"{} at once. The
branches can head in different directions and can later be merged
instead of always maintaing a central, \char`\"{}correct\char`\"{}
version. NTNU provided SVN and Trac repositories to keep track of our
project and we will use these to share finished code for evaluation
purposes. We decided to use Git due to its higher flexibility and because
of the experiences the team members already had with it.

\newpage
\section{Risk analysis and mitigation strategies}

This text will highlight different realistic risks that may occur during development
that can to some degree jeopardise the process or the final product.
A table with weighted risks and relative mitigation and remedial actions follows.

\begin{itemize}
\item Dropouts
\end{itemize}
The risk of people dropping the course or otherwise not being able to complete it as part
of the group. This can be caused by sickness as well.

\begin{itemize}
\item Arduino hardware
\end{itemize}
Our handed out Arduino equipment can fail, due to malfunctioning or wrong usage.
There is also the possibility that some of the hardware can be lost while we work with it at home.

\begin{itemize}
\item Deadlines
\end{itemize}
Throughout the course there is multiple deadlines that must be met. Failure to meet
these limits will have huge impacts on the grading and could possibly fail the group.

\begin{itemize}
\item Choosing wrong frameworks
\end{itemize}
We will necessarily have to build parts of our software around existing open source
frameworks to limit effort required by the task. If at a later point we have severe limitations
on our possibilities due to these frameworks the product could result poorer in features than
we originally planned. The impact can be negligible if other solutions are found.

\begin{itemize}
\item Design problems
\end{itemize}
During development features have to be constrained due to problems or resource limitations
which in turn will cause the final product to not satisfy the customer. If the features are
worked around and compromises can be found this will not have as huge of an impact.

\begin{itemize}
\item Wireless connectivity
\end{itemize}
If the Arduino chip modules (called shields) for Bluetooth etc. are too hard to implement
we would have to reconsider wireless connectivity as an option.
We set the 'get wireless to work' deadline to be one month. If we can't get it working
by that time we will have to use cabled connections instead and that would result clumsy
for a lot of concepts.

\begin{center}
\begin{tabular}{| l | l | l | l | p{2.8cm} | p{2.8cm} |}
\hline

Description & Likelihood & Impact & Risk & Mitigation & Remidial Action\\ \hline

Sickness 			& Med & Low & Med & Keep contagious sicknesses at home.
					& If the sickness is prolonged work tasks must be re-arranged appropriately. \\

Wireless Connect. & Low & High & Med & Get it working & Switch to cable connection \\

Arduino Malfun. & Low & Med & Low & Treat hardware properly. Do not eat or drink nearby.
					&  Get new hardware if possible.\\

Lost Hardware & Low & Med & Low & Have control over who has what and keep an inventory list.
					& Get new hardware if possible \\

Limited Product & Low & Med & Low & Thorough planning. Avoid late features implementation.
					&  Workaround problems at critical junctions in the process.\\

API Trouble & Med & Med & High & Limit the scope to documented open source APIs.
			& Investigate alternative solutions. Limit impact on productivity. \\

Final Deadline & Low & High & Med & Consistent work throughout the semester. Avoid last-minute feature implementation.
			&  Deliver the product in the best state possible.\\
Mid-Sem Deadline & Low & High & Med & Produce good documentation and begin early on reports.
			&  Consult with student assistants.\\

\hline
\end{tabular}
\end{center}

Concluding, most care should be put in the choice of existing frameworks and in early planning
and documentation efforts to avoid later problems related to the implementation and report delivery.

