\section{Conclusion}
\begin{itemize}
\item What did we learn?
\item Was the customer satisfied?
\item What pitfalls and problems did we encounter? (late meetings, bad hardware, etc.)
\item What would we do different in the future?
\item Specializing for Android vs portability.
\end{itemize}

\section{Further Work}
This section describes ideas, code and features we did not have time or resources to
finish at the project deadline. The section can also describe various interesting concepts 
that we visualized that we might have explored given more time.

\subsection{Supporting more communication technologies in ComLib}
Currently the ComLib only supports Bluetooth connection. Further work would be
to add support for additional communication technologies such as WiFi and Near Field 
Communication (NFC). The ComLib has been implemented so that this work should
be as easy as possible. Simply extend the Protocol class and implement the input and
output communication and the implementation should be fully forward and backwards
compatible with other versions of the ComLib.

\subsection{Supporting additional Social networks}
The SocialLib currently supports Twitter and Facebook. Support for additional social
networks like Google Plus, LinkedIn or MySpace was planned as future work. Also the
SocialLib has been implemented to accomodate this process as simple as possible
for the developer.

\subsection{Multi-Platform}
The Bluetooth part of the ComLib has been implemented using Android SDK. This means 
the Bluetooth part of the library will only run on an Android platform. The ComLib protcol
itself however, has been designed to support any type of platform. This means the ComLib
could be expanded to support other types of platform such as Bluetooth on iOS. This
should preferably be implemented as a common interface, so the developer only has to use
Bluetooth and the ComLib itself figures out if to use the Android version or the iOS version of
the Bluetooth Protocol.

\subsection{Security}
Currently the ComLib offers no level of security. Anyone with the mac address can connect
and have access to the full functionality of each prototype device. A future possibility could
be to define a security standard in the ComLib protocol like authentication with a password.